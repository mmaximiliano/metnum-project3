\section{Desarrollo}
\label{sec:desarrollo}
Como hemos contado, necesitamos simular el funcionamiento del tomógrafo
a partir de una imagen en escala de grises que consideramos como fuente de verdad. Para cumplir este objetivo debemos:

\begin{enumerate}
\item Levantar la imagen que queremos reconstruir
\item Definir un tamaño de discretización.
\item Generar un conjunto de rayos y calcular los tiempos de cada uno.
\item Agregar ruido a los tiempos obtenidos para modelar un problema real.
\item Obtener los valores inversos de las velocidades originales en cada celda.
\item Resolver la ecuación explicada en la sección \ref{eq:1} utilizando cuadrados mínimos % ver de agregar la referencia
\item Generar y guardar la imagen obtenida.
\end{enumerate}

A continuación vamos a explicar las decisiones más importantes tomadas a la hora de implementar la solución del problema en \verb|C++|.

\subsection{Simulación del tomógrafo computado}

\subsubsection{Procesamiento de imágenes}
Las imágenes que queremos reconstruir van a estar en formato \verb|PNG|. 
Para hacer la lectura de la imagen más sencilla utilizamos un script de python
 provisto por la cátedra para convertir imágenes en \verb|PNG| a \verb|CSV|. 
La imagen decidimos representarla como una \verb|matriz| de \verb|int|.

\subsubsection{Cálculo de distancias y tiempos}
% Modelado de rayos
Decidimos modelar los rayos con dos vectores bidimensionales,
que representan el punto de origen del rayo y el punto de destino,
en las coordenadas de la imagen original.

Luego, por cada rayo, vemos por qué píxeles de esta imagen pasa.
Tomamos el tiempo del rayo como la sumatoria de la intensidad de estos pixeles,
y contamos la distancia en cada uno de los sectores de la discretización
como la cantidad de pixeles del sector por los que pasa el rayo.

\subsubsection{Simulación de ruido}
\label{sec:desarrollo-noise}
% Uso de ruido normal con varianza como parametro
% Varianza elegida en proporcion a...?
Queremos modelar un instrumento del mundo real.
Dado que no existen instrumentos 100\% precisos y exactos,
vamos a agregar en nuestro modelo ruido aleatorio a los tiempos de los rayos.

Consideramos que lo más realista es utilizar una \textbf{distribución normal} con:
\begin{itemize}
	\item Media en 0
	\item Desvio en $\frac{p}{100} \times (255 \times n)$
\end{itemize}
Donde $p$ es un parámetro de entrada y $n$ es el tamaño de la imagen de salida.
En otras palabras, el ruido es una \textbf{variable aleatoria} con esperanza de no afectar en nada, y un desvío que es un 
porcentaje ($p$) a elegir del máximo tiempo posible ($255 \times n$)

Este porcentaje queda como parámetro para nuestro programa,
de forma de poder estudiar el comportamiento de los métodos utilizados con distintos niveles de ruido. 
Ademśs, si el tiempo de algún rayo queda menor a 1 luego de aplicar la distribución, 
lo cambiamos a 1 porque no tiene sentido tener tiempo negativos.

\subsection{Reconstrucción de la imagen}
Una vez simulados los tiempos y distancias,
tenemos el desafío de utilizarlos para reconstruir la imagen tomográfica.
Como contamos en la sección \ref{sec:intro-cml},
lo haremos aplicando \textbf{Cuadrados Mínimos Lineales} para minimizar

\begin{equation*}
	||Ds - t||_{2}
\end{equation*}

Y para esto primero calculamos la factorización \textbf{SVD} de \textit{D}.
Recordamos que la factorización SVD de $D \in \mathbb{R}^{mxn^{2}}$ es de la forma:

\begin{equation*}
	D = U \Sigma V^{t}
\end{equation*}

\subsubsection{Cuadrados Mínimos Lineales}
% Como se calcula a partir de SVD
% Complejidad
Dada la factorización SVD (de cuyo cálculo hablaremos en la sección \ref{sec:desarrollo-svd}),
seguimos el procedimiento explicado en la sección \ref{sec:intro-cml}.

Calculamos el vector $c = U^{t} t$.
Luego, como se demostró anteriormente, la solución de CML se da cuando el vector $z = V^{t} s$ cumple que:
\begin{equation*}
	z_{i} = c_{i}/\sigma_{i}
\end{equation*}

Entonces, calculamos z de esa manera. Finalmente, solo queda resolver el sistema $z = V^{t} s$,
el cual es simple ya que \textit{V} es ortogonal.
Es decir, podemos computar s de la siguiente manera:
\begin{equation*}
	s = V z
\end{equation*}

Dado que los datos tienen ruido, al reconstruir la imagen con CML
quedan algunos datos fuera del rango posible para un píxel
(0 a 255, dado que es escala de grises).
Tomamos la decisión de saturar estos valores.
Es decir, en caso de que un píxel de negativo, le damos el valor 0,
y en caso de quedar con un valor mayor a 255, le asignamos 255.

\subsubsection{Descomposición en SVD}
\label{sec:desarrollo-svd}
% Problema numerico (no encontrar todos los autovalores)
% Esto nos hizo plantear un experimento al respecto
% Tardanza en el computo, complejidad
Para obtener la factorización, requerimos calcular los autovalores de $D^{t} D$.
Estos los obtenemos realizando la multiplicación
y aplicando repetidas veces el \textbf{Método de la Potencia} con \textbf{deflación}\cite{pot}.
Este nos sirve para calcular tanto los autovalores como los autovectores de $D^{t} D$.
Esto es especialmente útil, dado que estos autovectores sirven como columnas de la matriz \textit{V}.
Más aún, dado que \textit{V} es ortogonal y $\Sigma$ es diagonal, podemos obtener U fácilmente despejando:

\begin{equation*}
	D V \Sigma^{-1} = U
\end{equation*}

Para el Método de la Potencia, utilizamos como criterio de corte 100 iteraciones.
Esto es debido a que encontramos imprecisiones utilizando menos iteraciones,
y dado que realizaremos una gran cantidad de veces deflación (arrastrando los errores del Método de la Potencia),
preferimos utilizar un criterio conservador. Esto lo pudimos ver a través de los tests que realizamos para \verb|SVD| con imágenes con las que vamos a experimentar en la sección \ref{sec:experimentacion}. Una buena mejora en el trabajo práctico sería experimentar sobre la decisión de este corte. Pero dado el tiempo limitado y que pudimos ver a través de tests que 100 es un buen número, tomamos la decisión de usar este número como criterio de corte y experimentar sobre cosas más primordiales.

No encontramos mayores dificultades en el desarrollo del método.
Sin embargo, a la hora de probarlo con tests,
nos encontramos con dos inconvenientes relacionados al cálculo de SVD.

El primero fue que el cálculo de SVD tardaba mucho más que el resto de métodos.
Analizando más en detalle, vimos que esto se daba en dos partes:
\begin{itemize}
	\item Calculando $D^{t} D$: esto se debe a que, siendo $D \in \mathbb{R}^{mxn^{2}}$,
		esta multiplicación cuesta $O(m * (n^2)^2)$.
	\item Haciendo el Método de la Potencia con deflación:
		cada iteración del Método de la Potencia cuesta $O(n^2 * 100)$ operaciones
		(multiplicando $D^{t} D \in \mathbb{R}^{n^{2}xn^{2}}$ por un vector).
		Luego, cada vez que realizamos deflación, cuesta $O(n^{2})$.
		Haciendo esto para cada uno de los $n^2$ autovalores, cuesta $O(n^2 n^2) = O(n^4)$.
\end{itemize}

Esto nos llevó a realizar optimizaciones, las cuales explicaremos mejor en la sección \ref{sec:desarrollo-opt}.

El segundo, que nos encontramos con que el Método de la Potencia,
luego de varias aplicaciones de deflación, comenzaba a devolver autovalores negativos cercanos a 0.
Esto no tiene sentido teórico, ya que $D^{t} D$ es una matriz simétrica semidefinida positiva,
por lo que intuimos que se debe dar por inestabilidad numérica del Método de la Potencia con deflación.
Sin embargo, estos autovalores son los de menor valor absoluto.
Como vimos en la sección \ref{sec:lowrank}, podemos descartarlos y truncar la descomposición,
perdiendo relativamente poca información.

Esta idea nos resultó interesante,
dado que el cálculo de autovalores es una de las partes que más costo computacional tiene.
Es por eso que nos propusimos experimentar limitando la cantidad de autovalores que calculamos,
para ver la relación entre el tiempo ahorrado y la pérdida de calidad en las reconstrucciones.

\subsection{Optimizaciones}
\label{sec:desarrollo-opt}
% Cache y threads
\subsubsection{Reutilización de SVD}
Al cambiar la intensidad del ruido, manteniendo el resto de parámetros idénticos,
la matriz \textit{D} se mantiene igual.
Por lo tanto, vimos la oportunidad de guardar en un archivo la factorización SVD la primera vez que la calculamos.
De esta forma, las siguientes veces que realicemos el mismo experimento con distinto ruido,
lo único que hacemos es recuperar la factorización de este archivo.
Por motivos obvios, esto \textbf{no} es usado en los experimentos
que tienen como objetivo medir la velocidad de los métodos.

\subsubsection{Paralelización de cálculos}
Tanto el cálculo de $D^{t} D$ como el Método de la Potencia
usan repetidas veces la multiplicación entre matriz y vector,
una operación que resulta trivial de paralelizar.
Es por eso que a este operador le agregamos paralelización utilizando \textit{threads}.
