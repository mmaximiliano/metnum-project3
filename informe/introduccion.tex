\section{Introducción}
\label{sec:introduccion}
% Escribir contexto del problema, qué es una tomografía computada y por qué nos interesa
Cada vez se utilizan herramientas computacionales en más ramas de las ciencias, incluyendo a las ciencias médicas.
Las aplicaciones de sistemas asistidos por computadoras en distintas especialidades de estas ciencias
presentan una gran mejoría en el sistema de salud a nivel mundial\footnote{https://es.wikipedia.org/wiki/Inform\%C3\%A1tica\_en\_salud}.
El veloz avance de la computación (tanto en software como en hardware)
posibilita un constante progreso en el desarrollo de nuevas aplicaciones médicas.
Algunos de los primeros usos fueron el procesamiento de historias clínicas y datos epidemiológicos.
Luego se han agregado los instrumentos de diagnóstico y tratamiento médicos, entre otros\footnote{http://www.cocmed.sld.cu/no63/n63edi.htm}.

En el presente trabajo, nos interesa estudiar la aplicación de diferentes técnicas algorítmicas
y de aproximación para la reconstrucción de imágenes.
En particular, desarrollaremos un programa que nos permita realizar reconstrucciones de imágenes tomográficas computadas.
Estas consisten en atravesar el objeto de estudio con rayos X,
para así lograr estimar la densidad del mismo.
Las imágenes tomográficas computadas permiten analizar las estructuras internas de las distintas partes del organismo,
lo cual facilita el diagnóstico de distintos problemas médicos, como fracturas, hemorragias internas, tumores o infecciones en los distintos órganos.
También permite conocer la morfología de la médula espinal y de los discos intervertebrales
o medir la densidad ósea, importante en el caso de osteoporosis\footnote{https://www.nibib.nih.gov/science-education/science-topics/computed-tomography-ct}.

Un problema que se presenta, es que los instrumentos utilizados poseen errores de medición.
Y esto debe ser tenido en cuenta, pues por los usos descritos anteriormente un error de este tipo
podría derivar en un mal diagnóstico,
lo cual conllevaría consecuencias aún peores.
Esto nos motiva a estudiar diferentes métodos de aproximación para solventar los problemas de precisión.
En particular, analizaremos el método de cuadrados mínimos lineales.
Veremos la calidad de las reconstrucciones que consigue y la velocidad a la cual lo hace,
además de experimentar con distintos tipos de tomografías y niveles de ruido, entre otras cosas.